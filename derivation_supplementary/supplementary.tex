% Created 2016-06-19 Sun 00:51
\documentclass[colorlinks]{article}
\usepackage[utf8]{inputenc}
\usepackage[T1]{fontenc}
\usepackage{amsmath, amssymb}
\usepackage{hyperref}
%% \tolerance=1000
\hypersetup{allcolors = blue} % to have all the hyperlinks in 1 color
\renewcommand{\refname}{\vspace{-.8cm}}
\usepackage{jneurosci}
\bibliographystyle{jneurosci}
\author{Y. Zerlaut \& A. Destexhe}
\date{\today}
\title{Supplementary Material \\ \large{Heterogenous firing responses leads to diverse coupling to presynaptic activity in a simplified morphological model of layer V pyramidal neurons}}
\begin{document}

\maketitle
\tableofcontents

\newpage

\section{Reduction to the equivalent cylinder}
\label{sec-1}
\label{sec:reduction-to-eq-cyl}

The key of the derivation relies on having the possibility to reduce
the complex morphology to an equivalent cylinder \cite{Rall1962}. We
adapted this procedure to capture the change in integrative properties
of the membrane that results from the mean synaptic bombardment during
active cortical states, reviewed in \citetext{Destexhe2003}.

For a set of synaptic stimulation \(\{\nu^e_p, \nu_i^p, \nu_e^d,
\nu_i^d, s\}\), let's introduce the following stationary densities of
conductances:

\begin{equation}
\left\{
\begin{split}
& g_{e0}^p = \pi \, d \, \mathcal{D}_e \, \nu_e^p \, \tau_e^p \, Q_e^p \quad ;  \quad
g_{i0}^p = \pi \, d \, \mathcal{D}_i \, \nu_i^p \, \tau_i^p \, Q_i^p \\
& g_{e0}^d = \pi \, d \, \mathcal{D}_e  \, \nu_e^d \, \tau_e^d \, Q_e^d \quad ; \quad 
g_{i0}^d = \pi \, d \, \mathcal{D}_i \, \nu_i^d \, \tau_i^d \, Q_i^d
\end{split}
\right.
\label{eq:1}
\end{equation}

where $\mathcal{D}_e$ and $\mathcal{D}_i$ are the excitatory and
inhibitory synaptic densities.

We introduce two activity-dependent electrotonic constants relative to
the proximal and distal part respectively:

\begin{equation}
\lambda^p= \sqrt{\frac{r_m}{r_i (1 + r_m g_{e0}^p + r_m g_{i0}^p)}}
\quad 
\lambda^d= \sqrt{\frac{r_m}{r_i (1 + r_m g_{e0}^d + r_m g_{i0}^d)}}
\end{equation}

For a dendritic tree of total length $l$, whose proximal part ends at
$l_p$ and with $B$ evenly spaced generations of branches, we define
the space-dependent electrotonic constant:

\begin{equation}
 \lambda(x) = \big( \lambda^p + \mathcal{H}(x-l_p)(\lambda^d-\lambda^p) \big)
 2^{- \frac{1}{3} \, \lfloor \frac{B \, x}{l} \rfloor}
\end{equation}

where \( \lfloor . \rfloor \) is the floor function. Note that
$\lambda(x)$ is constant on a given generation, but it decreases from
generation to generation because of the decreasing diameter along the
dendritic tree. It also depends on the synaptic activity and
therefore has a discontinuity at $x=l_p$.

Following \citetext{Rall1962}, we now define a dimensionless length $X$:

\begin{equation}
 X(x) = \int_0^x \frac{dx}{\lambda(x)}
\label{eq:x-rescale}
\end{equation}

We define \(L= X(l) \) and \(L_p= X(l_p) \), the total length and
proximal part length respectively (capital letters design rescaled
quantities).

\section{Mean membrane potential}
\label{sec-2}
\label{sec:mean-derivation}

We derive the mean membrane potential \(\mu_V(x)\) corresponding to
the stationary response to constant densities of conductances given by
the means of the synaptic stimulation. We obtain the stationary
equations by removing temporal derivatives in Equation, the set of
equation governing this mean membrane potential in all branches is
therefore:

\begin{equation}
\label{eq:model-equation-muV}
\left\{
\begin{split}
& \frac{1}{r_i} \frac{\partial^2 \mu_v}{\partial x^2} = \frac{\mu_v(x)-E_L}{r_m} \\
& \qquad - g_{e0}^p \, (\mu_v(x) - E_e) - g_{0i}^p \, (\mu_v(x) - E_i) \quad \forall x \in [0,l_p] \\
& \frac{1}{r_i} \frac{\partial^2 \mu_v}{\partial x^2} =  \frac{\mu_v(x)-E_L}{r_m} \\
& \qquad - g_{e0}^d \, (\mu_v(x) - E_e) - g_{0i}^d \, (\mu_v(x) - E_i)  \quad \forall x \in [l_p,l]  \\
& \frac{\partial \mu_v}{\partial x}_{|x=0} = r_i \, \big(
 \frac{\mu_v(0)-E_L}{R_m} + G_{i0}^S \, ( \mu_v(0)-E_i) \big)\\
& \mu_v(l_p^-,t) = \mu_v(l_p^+,t) \\
& \frac{\partial \mu_v}{\partial x}_{l_p^-} 
= \frac{\partial \mu_v}{\partial x}_{l_p^+} \\
& \frac{\partial \mu_v}{\partial x}_{x=l} = 0
\end{split}
\right.
\end{equation}

Because the reduction to the equivalent cylinder conserves the
membrane area and the previous equation only depends on density of
currents, the equation governing $\mu_v(x)$ in all branches can be
transformed into an equation on an equivalent cylinder of length $L$.
We rescale $x$ by $\lambda(x)$ (see Equation \ref{eq:x-rescale}) and we
obtain the equation verified by \(\mu_V(X)\):

\begin{equation}
\label{eq:model-equation-muV-rescaled}
\left\{
\begin{split}
& \frac{\partial^2 \mu_v}{\partial X^2} = \mu_v(X)-v_0^p \quad \forall X \in [0,L_p] \\
&\frac{\partial^2 \mu_v}{\partial X^2} = \mu_v(X)-v_0^d \quad \forall X \in [L_p,L]  \\
&\frac{\partial \mu_v}{\partial X}|_{X=0} = \gamma^p \, 
\Big( \mu_v(0) - V_0  \Big) \\
&\mu_v(L_p^-) = \mu_v(L_p^+) \\
&\frac{\partial \mu_v}{\partial X}_{L_p^-} = \frac{\lambda^p}{\lambda^d}
\frac{\partial \mu_v}{\partial X}_{L_p^+} \\
&\frac{\partial \mu_v}{\partial X}_{X=L} = 0 
\end{split}
\right.
\end{equation}

where:

\begin{equation}
\begin{split}
& v_0^p = \frac{E_L + r_m g_{e0}^p E_e + r_m g_{i0}^p E_i}{ 1 + r_m g_{e0}^p + r_m g_{i0}^p }\\
& v_0^d = \frac{E_L + r_m g_{e0}^d E_e + r_m g_{i0}^d E_i}{ 1 + r_m g_{e0}^d + r_m g_{i0}^d }\\
& \gamma^p = \frac{r_i \lambda^p \, (1+ G_i^0 R_m)}{R_m}\\
& V_0 = \frac{E_L + G_i^0 R_m E_i}{ 1 +  + G_i^0 R_m }
\end{split}
\end{equation}

We write the solution on the form:

\begin{equation}
\left\{
\begin{split}
& \mu_v(X) = v_0^p + A \, \cosh(X) + C \, \sinh(X) \quad \forall \, X \in [0,L_p] \\
& \mu_v(X) = v_0^d + B \, \cosh(X-L) + D \, \sinh(X-L) \quad \forall \, X \in [L_pL]
\end{split}
\right.
\label{eq:muV-final}
\end{equation}


\begin{itemize}
\item Sealed-end boundary condition at cable end implies $D=0$

\item Somatic boudary condition imply: $C = \gamma^p \, (v_0^p - V_0 + A)$

\item Then v continuity imply :
\( v_0^p + A \, \cosh(L_p) +  \gamma^p \, (v_0^p - V_0 + A) \, \sinh(L_p) = v_0^d + B \, \cosh(L_p-L) \)

\item Then current conservation imply: 
\( A \, \sinh(L_p) +  \gamma^p \, (v_0^p - V_0 + A) \, \cosh(L_p) = \frac{\lambda^p}{\lambda^d} \, B \, \sinh(L_p-L) \)
\end{itemize}

We rewrite those condition on a matrix form:

\begin{equation}
\Big(
\begin{matrix}
    \cosh(L_p)+\gamma^p \sinh(L_p) & -\cosh(L_p-L) \\
    \sinh(L_p)+\gamma^p \cosh(L_p) & -  \frac{\lambda^p}{\lambda^d} \, \sinh(L_p-L) 
\end{matrix}
\Big)
\cdot
\Big(
\begin{matrix}
    A \\
    B 
\end{matrix}
\Big) = 
\Big(
\begin{matrix}
v_0^d - v_0^p - \gamma^p \, (v_0^p-V_0) \, \sinh(L_p) \\
- \gamma^p \, (v_0^p-V_0) \, \cosh(L_p)
\end{matrix}
\Big)
\end{equation}

And we solved this equation with the \texttt{solve\_linear\_system\_LU} method of \texttt{sympy}

The coefficients $A$ and $B$ are given by:
\begin{equation}
A=\frac{\alpha}{\beta} \qquad \qquad B=\frac{\gamma}{\delta}
\end{equation}

where:

\begin{equation}
\begin{split}
& \alpha = V_{0} \gamma^{P} \lambda^{D} \cosh{\left (L_{p} \right )}
\cosh{\left (L - L_{p} \right )} + V_{0} \gamma^{P} \lambda^{P}
\sinh{\left (L_{p} \right )} \sinh{\left (L - L_{p} \right )} \\
& \quad - \gamma^{P} \lambda^{D} v^{d}_{0} \cosh{\left (L_{p} \right )}
\cosh{\left (L - L_{p} \right )} - \gamma^{P} \lambda^{P} v^{d}_{0}
\sinh{\left (L_{p} \right )} \sinh{\left (L - L_{p} \right )} \\
& \quad - \lambda^{P} v^{d}_{0} \sinh{\left (L - L_{p} \right )}
 + \lambda^{P} v^{p}_{0} \sinh{\left (L - L_{p} \right )} \\
& \beta = \gamma^{P} \lambda^{D} \cosh{\left (L_{p} \right )}
\cosh{\left (L - L_{p} \right )} + \gamma^{P} \lambda^{P} \sinh{\left
(L_{p} \right )} \sinh{\left (L - L_{p} \right )} + \\
& \quad \lambda^{D} \sinh{\left (L_{p} \right )} \cosh{\left (L - L_{p} \right )}
 + \lambda^{P} \sinh{\left (L - L_{p} \right )} \cosh{\left (L_{p} \right
)} \\
& \gamma = \lambda^{D} \big( V_{0} \gamma^{P} + \gamma^{P} v^{d}_{0}
\cosh{\left (L_{p} \right )} - \gamma^{P} v^{d}_{0} \\
& \quad  - \gamma^{P}
v^{p}_{0} \cosh{\left (L_{p} \right )} + v^{d}_{0} \sinh{\left (L_{p}
\right )} - v^{p}_{0} \sinh{\left (L_{p} \right )} \big) \\
& \delta = \gamma^{P}
\lambda^{D} \cosh{\left (L_{p} \right )} \cosh{\left (L - L_{p} \right
)} + \gamma^{P} \lambda^{P} \sinh{\left (L_{p} \right )} \sinh{\left
(L - L_{p} \right )}  \\
& \quad + \lambda^{D} \sinh{\left (L_{p} \right )}
\cosh{\left (L - L_{p} \right )}  + \lambda^{P} \sinh{\left (L - L_{p}
\right )} \cosh{\left (L_{p} \right )}
\end{split}
\end{equation}

\section{Membrane potential response to a synaptic event}
\label{sec-3}
\label{sec:eq-for-var-around-mean}

We now look for the response to \(n_{src}= \lfloor \frac{B \,
x_{src}}{l} \rfloor} \) synaptic events at position \(x_{src}\) on all
branches of the generation of \(x_src\), those events have a
conductance $g(t)/n_{src}$ and reversal potential $E_{rev}$. We make
the hypothesis that the initial condition correspond to the stationary
mean membrane potential $\mu_V(x)$. This potential will also be used
to fix the driving force at the synapse to $\mu_v(x_{src})-E_{rev}$,
this linearizes the equation and will allow an analytical
treatment. To derive the equation for the response around the mean
\(\mu_v(x)\), we rewrite Equation 9 in main text with \(v(x, t) =
\delta v(x, t) + \mu_v(x)\), we obtain the equation for \(\delta v(x,
t)\):

\begin{equation}
\label{eq:psp-equation}
\left\{
\begin{split}
& \frac{1}{r_i} \frac{\partial^2 \delta v}{\partial x^2} = 
c_m \frac{\partial \delta v}{\partial t} 
+ \frac{\delta v}{r_m} (1 + r_m \, g_{e0}^p + r_m \, g_{i0}^p) \\
& \qquad - \delta(x-x_{src}) \, \big(\mu_v(x_{src})-E_{rev}\big) \, \frac{g(t)}{n_{src}},
\quad \forall x \in [0,l_p] \\
& \frac{1}{r_i} \frac{\partial^2 \delta v}{\partial x^2} = 
c_m \frac{\partial \delta v}{\partial t} 
+ \frac{\delta v}{r_m} (1 + r_m \, g_{e0}^d + r_m \, g_{i0}^d) \\
& \qquad
- \delta(x-x_{src}) \, \big(\mu_v(x_{src})-E_{rev}\big) \, \frac{g(t)}{n_{src}},
\quad   \forall x \in [l_p,l]\\
& \frac{1}{r_i} \frac{\partial \delta v}{\partial x}_{|x=0} = 
 C_M \frac{\partial \delta  v}{\partial t}_{|x=0} +  
 \frac{ \delta v(0,t)}{R_m} ( 1+ R_m G_{i0}^S) \\
&  \delta v(l_p^-,t) = \delta v(l_p^+,t) \\
& \frac{\partial \delta v}{\partial x}_{l_p^-} 
= \frac{\partial \delta v}{\partial x}_{l_p^+} \\
& \frac{\partial \delta v}{\partial x}_{x=l} = 0
\end{split}
\right.
\end{equation}

Because this synaptic event is concomitant in all branches at distance
$x_{src}$, we can use again the reduction to the equivalent cylinder
(note that the event has now a weight multiplied by $n_{src}$ so that
its conductance becomes \(g(t)\)), we obtain:

\begin{equation}
\label{eq:psp-equation-reduced}
\left\{
\begin{split}
& \frac{\partial^2 \delta v}{\partial X^2} =
\big( \tau_m^p + (\tau_m^d-\tau_m^p) 
\mathcal{H}(X-L_p) \big) \frac{\partial \delta v}{\partial t} + \delta v \\
& \qquad - \big(\mu_v(X_{src})-E_{rev}\big) \,  \delta(X-X_{src}) \times \\
& \qquad \qquad \frac{g(t)}{c_m} 
  \big( \frac{\tau_m^p}{\lambda^p} +
  (\frac{\tau_m^d}{\lambda^d}-\frac{\tau_m^p}{\lambda^p}) 
\mathcal{H}(X_{src}-L_p) \big) \\
& \frac{\partial \delta v}{\partial X}_{|X=0} = 
 \gamma^p \big( \tau_m^S \frac{\partial \delta  v}{\partial t}_{|X=0} +  
 \delta v(0,t) \big) \\
& \delta v(L_p^-,t) = \delta v(L_p^+,t) \\
& \frac{\partial \delta v}{\partial X}_{L_p^-} 
= \frac{\lambda^p}{\lambda^d} \, 
\frac{\partial \delta v}{\partial X}_{L_p^+} \\
& \frac{\partial \delta v}{\partial X}_{X=L} = 0
\end{split}
\right.
\end{equation}

where we have introduced the following time constants:

\begin{equation}
\label{eq:time-constants}
\begin{split}
 & \tau_m^D = \frac{r_m \, c_m}{1+r_m \, g_{e0}^d+r_m \, g_{i0}^d} \\
 & \tau_m^P = \frac{r_m \, c_m}{1+r_m \, g_{e0}^p+r_m \, g_{i0}^p} \\
 & \tau_m^S = \frac{R_m \, C_m}{1+R_m \, G_{i0}^S}
\end{split}
\end{equation}

We now use distribution theory (see \citetext{Appel2008} for a
comprehensive textbook) to translate the synaptic input into boundary
conditions at $X_{src}$, physically this corresponds to: 1) the
continuity of the membrane potential and 2) the discontinuity of the
current resulting from the synaptic input.

\begin{equation}
\left\{
\begin{split}
&  \delta v(X_{src}^-,f) = \delta v(X_{src}^+,f) \\
& \frac{\partial \delta v}{\partial X}_{X_{src}^+} 
- \frac{\partial \delta v}{\partial X}_{X_{src}^-} 
= - \big(\mu_v(X_{src})-E_{rev}\big) \times \\
& \qquad \qquad \qquad 
\big( \frac{\tau_m^p}{\lambda^p} +
  (\frac{\tau_m^d}{\lambda^d}-\frac{\tau_m^p}{\lambda^p}) 
 \mathcal{H}(X_{src}-L_p) \big) \, \frac{g(t)}{c_m} \\
\end{split}
\right.
\end{equation}

We will solve Equation \ref{eq:psp-equation-reduced} by using Fourier
analysis. We take the following convention for the Fourier transform:

\begin{equation}
\label{eq:fourier-convention}
\hat{F}(f) = \int_\mathbb{R} F(t) \, e^{- 2 i \pi f t} \, dt
\end{equation}

We Fourier transform the set of Equations \ref{eq:psp-equation-reduced}, we
obtain:

\begin{equation}
\label{eq:psp-equation-fourier}
\left\{
\begin{split}
& \frac{\partial^2 \hat{\delta v}}{\partial X^2} =
\big( \alpha_f^p + (\alpha_f^d-\alpha_f^p) 
\mathcal{H}(X-L_p) \big)^2 \, 
\hat{\delta v}  \\
& \frac{\partial \hat{\delta v}}{\partial X}_{|X=0} = 
 \gamma_f^p \,  \hat{\delta v}(0,f) \\
&  \hat{\delta v}(X_{src}^-,f) = \hat{\delta v}(X_{src}^+,f) \\
& \frac{\partial \hat{\delta v}}{\partial X}_{X_{src}^-} 
= \frac{\partial \hat{\delta v}}{\partial X}_{X_{src}^+} 
- \big(\mu_v(X_{src})-E_{rev}\big) \times \\
& \qquad \qquad \qquad \qquad 
\big( r_f^p + (r_f^d-r_f^p) \mathcal{H}(X_{src}-L_p) \big) \, \hat{g(f)} \\
& \hat{\delta v}(L_p^-,f) = \hat{\delta v}(L_p^+,f) \\
& \frac{\partial \hat{\delta v}}{\partial X}_{L_p^-} 
= \frac{\lambda^p}{\lambda^d} \, 
\frac{\partial \hat{\delta v}}{\partial X}_{L_p^+} \\
& \frac{\partial \hat{\delta v}}{\partial X}_{X=L} = 0
\end{split}
\right.
\end{equation}

where 
\begin{equation}
\begin{split}
& \alpha_f^p = \sqrt{1+ 2 i \pi f \tau_m^p} \qquad
 r_f^p = \frac{\tau_m^p}{c_m \, \lambda^p} \\
& \alpha_f^d = \sqrt{1+ 2 i \pi f \tau_m^d} \qquad
 r_f^d = \frac{\tau_m^d}{c_m \, \lambda^d} \\
& \qquad \qquad \gamma_f^p =  \gamma^p \, (1+ 2 i \pi f \tau_m^S) \\
\end{split}
\end{equation}

To obtain the solution, we need to split the solution into two cases:

\begin{enumerate}
\item $X_{src} \leq L_p$
\label{sec-3-0-0-1}

Let's write the solution to this equation as the form (already
including the boundary conditions at $X=0$ and $X=L$):

\begin{equation}
\begin{split}
& \hat{\delta v}(X, X_{src}, f) = \\
&\left\{
\begin{split}
& A_f(X_{src}) \, \big ( \cosh(\alpha_f^p \, X)+\gamma^p \, \sinh(\alpha_f^p \, X) \big) \\
& \qquad \qquad \mathrm{ if: } 0 \leq X \leq X_{src} \leq L_p \leq L \\
& B_f(X_{src})\, \cosh(\alpha_f^p \, (X-L_p))+C_f(X_{src})\, \sinh(\alpha_f^p \, (X-L_p)) \\
& \qquad \qquad \mathrm{ if: } 0 \leq X_{src}  \leq X \leq L_p \leq L \\
& D_f(X_{src}) \, \cosh(\alpha_f^d \, (X-L) )  \\
& \qquad \qquad \mathrm{ if: } 0 \leq X_{src} \leq L_p  \leq X \leq L 
\end{split}
\right.
\end{split}
\end{equation}

We write the 4 conditions correspondingto the conditions in $X_{src}$
and $L_p$ to get $A_f, B_f, C_f, D_f$. On a matrix form, this gives:

\begin{equation}
\hspace{-4cm}
\footnotesize
M = 
\begin{pmatrix}
    \cosh(\alpha_f^p \, X_{src})+\gamma_f^p \sinh(\alpha_f^p \, X_{src})  
         & -\cosh(\alpha_f^p \, (X_{src}-L_p)) & -\sinh(\alpha_f^p \, (X_{src}-L_p)) & 0 \\
    \alpha_f^p \big ( \sinh(\alpha_f^p \, X_{src})+\gamma_f^p \cosh(\alpha_f^p \, X_{src})  \big)
         & - \alpha_f^p \sinh(\alpha_f^p \, (X_{src}-L_p)) & -\alpha_f^p \cosh(\alpha_f^p \, (X_{src}-L_p)) & 0 \\
    0 & 1 & 0 & - \cosh(\alpha_f^d \, (L_p-L)) \\
    0 & 0 & \alpha_f^p & -  \alpha_f^d \frac{\lambda^p}{\lambda^d} \, \sinh(\alpha_f^d \, (L_p-L))
\end{pmatrix}
\end{equation}

\begin{equation}
M \cdot
\begin{pmatrix}
    A_f \\
    B_f \\
    C_f \\
    D_f
\end{pmatrix} = 
\begin{pmatrix}
0 \\
- r_f^p I_f \\
0 \\
0
\end{pmatrix}
\end{equation}

And we will solve it with the \texttt{solve\_linear\_system\_LU} method of
\texttt{sympy}. For the $A_f(X_{src})$ coefficient, we obtain:

\begin{equation}
A_f(X_{src}) = \frac{a^1_f(X_{src})}{a^2_f(X_{src})}
\end{equation}

with:

\begin{equation}
\begin{split}
& a^1_f(X_{src}) = I_{f} r^{P}_{f} \left(- \alpha^{D}_{f} \lambda^{P} \cosh{\left (L \alpha^{D}_{f} - L_{p} \alpha^{D}_{f} - L_{p} \alpha^{P}_{f} + X_{s} \alpha^{P}_{f} \right )}  \\
& \qquad + \alpha^{D}_{f} \lambda^{P} \cosh{\left (L \alpha^{D}_{f} - L_{p} \alpha^{D}_{f} + L_{p} \alpha^{P}_{f} - X_{s} \alpha^{P}_{f} \right )}  \\
& \qquad + \alpha^{P}_{f} \lambda^{D} \cosh{\left (L \alpha^{D}_{f} - L_{p} \alpha^{D}_{f} - L_{p} \alpha^{P}_{f} + X_{s} \alpha^{P}_{f} \right )}  \\
& \qquad + \alpha^{P}_{f} \lambda^{D} \cosh{\left (L \alpha^{D}_{f} - L_{p} \alpha^{D}_{f} + L_{p} \alpha^{P}_{f} - X_{s} \alpha^{P}_{f} \right )}\right)\\
& a^2_f(X_{src}) = \alpha^{P}_{f} \left(- \alpha^{D}_{f} \gamma^{P}_{f} \lambda^{P} \cosh{\left (- L \alpha^{D}_{f} + L_{p} \alpha^{D}_{f} + L_{p} \alpha^{P}_{f} \right )}  \\
& \qquad + \alpha^{D}_{f} \gamma^{P}_{f} \lambda^{P} \cosh{\left (L \alpha^{D}_{f} - L_{p} \alpha^{D}_{f} + L_{p} \alpha^{P}_{f} \right )} -  \\
& \qquad \alpha^{D}_{f} \lambda^{P} \sinh{\left (- L \alpha^{D}_{f} + L_{p} \alpha^{D}_{f} + L_{p} \alpha^{P}_{f} \right )}  \\
& \qquad + \alpha^{D}_{f} \lambda^{P} \sinh{\left (L \alpha^{D}_{f} - L_{p} \alpha^{D}_{f} + L_{p} \alpha^{P}_{f} \right )}  \\
& \qquad + \alpha^{P}_{f} \gamma^{P}_{f} \lambda^{D} \cosh{\left (- L \alpha^{D}_{f} + L_{p} \alpha^{D}_{f} + L_{p} \alpha^{P}_{f} \right )}  \\
& \qquad + \alpha^{P}_{f} \gamma^{P}_{f} \lambda^{D} \cosh{\left (L \alpha^{D}_{f} - L_{p} \alpha^{D}_{f} + L_{p} \alpha^{P}_{f} \right )}  \\
& \qquad + \alpha^{P}_{f} \lambda^{D} \sinh{\left (- L \alpha^{D}_{f} + L_{p} \alpha^{D}_{f} + L_{p} \alpha^{P}_{f} \right )}  \\
& \qquad + \alpha^{P}_{f} \lambda^{D} \sinh{\left (L \alpha^{D}_{f} - L_{p} \alpha^{D}_{f} + L_{p} \alpha^{P}_{f} \right )}\right)\\
\end{split}
\end{equation}

\item $ L_p \le X_{\text{src}}$
\label{sec-3-0-0-2}


Let's write the solution to this equation as the form (already
including the boundary conditions at $X=0$ and $X=L$:

\begin{equation}
\begin{split}
& \hat{\delta v}(X, X_{src}, f) = \\
&\left\{
\begin{split}
& E_f(X_{src}) \, \big ( \cosh(\alpha_f^p \, X)+\gamma^p \, \sinh(\alpha_f^p \, X) \big) \\
& \qquad \qquad \mathrm{ if: } 0 \leq X \leq L_p \leq X_{src} \leq L \\
& F_f(X_{src})\, \cosh(\alpha_f^d \, (X-L_p))+G_f(X_{src})\, \sinh(\alpha_f^d \, (X-L_p)) \\
& \qquad \qquad \mathrm{ if: } 0  \leq L_p  \leq X \leq X_{src} \leq L \\
& H_f(X_{src}) \, \cosh(\alpha_f^d \, (X-L) )  \\
& \qquad \qquad \mathrm{ if: } 0  \leq L_p \leq X_{src}  \leq X \leq L 
\end{split}
\right.
\end{split}
\end{equation}

We write the 4 conditions correspondingto the conditions in $X_{src}$
and $L_p$ to get $A_f, B_f, C_f, D_f$. On a matrix form, this gives:

We rewrite this condition on a matrix form:

\begin{equation}
\hspace{-4cm}
\footnotesize
M_2 = 
\begin{pmatrix}
    \cosh(\alpha_f^p \, L_p)+\gamma_f^p \sinh(\alpha_f^p \, L_p) & -1 & 0 &0 & 0 \\
    \alpha_f^p \big ( \sinh(\alpha_f^p \, L_p)+\gamma_f^p \cosh(\alpha_f^p \, L_p)  \big) 
         & 0 & -  \alpha_f^d \frac{\lambda^p}{\lambda^d} & 0 \\
    0 & \cosh(\alpha_f^d \, (X_{src}-L_p)) & \sinh(\alpha_f^d \, (X_{src}-L_p)) & - \cosh(\alpha_f^d \, (X_{src}-L))\\
    0 & \alpha_f^d \, \sinh(\alpha_f^d \, (X_{src}-L_p)) & \alpha_f^d \, \cosh(\alpha_f^d \, (X_{src}-L_p))
         & - \alpha_f^d \, \sinh(\alpha_f^d \, (X_{src}-L))\\
\end{pmatrix}
\end{equation}

\begin{equation}
M \cdot
\begin{pmatrix}
    E_f \\
    F_f \\
    G_f \\
    H_f
\end{pmatrix} = 
\begin{pmatrix}
0 \\
0 \\
0 \\
- r_f^d I_f
\end{pmatrix}
\end{equation}

And we will solve it with the \texttt{solve\_linear\_system\_LU} method of
\texttt{sympy}. For the $E_f(X_{src})$ coefficient, we obtain:

\begin{equation}
E_f(X_{src}) = \frac{e^1_f(X_{src})}{e^2_f(X_{src})}
\end{equation}

with:

\begin{equation}
\begin{split}
& e^1_f(X_{src}) = 2 I_{f} \lambda^{P} r^{D}_{f} \cosh{\left (\alpha^{D}_{f} \left(L - X_{s}\right) \right )} \\
& e^2_f(X_{src}) = - \alpha^{D}_{f} \gamma^{P}_{f} \lambda^{P} \cosh{\left (- L \alpha^{D}_{f} + L_{p} \alpha^{D}_{f} + L_{p} \alpha^{P}_{f} \right )} \\
& \qquad + \alpha^{D}_{f} \gamma^{P}_{f} \lambda^{P} \cosh{\left (L \alpha^{D}_{f} - L_{p} \alpha^{D}_{f} + L_{p} \alpha^{P}_{f} \right )}  \\
& \qquad- \alpha^{D}_{f} \lambda^{P} \sinh{\left (- L \alpha^{D}_{f} + L_{p} \alpha^{D}_{f} + L_{p} \alpha^{P}_{f} \right )}  \\
& \qquad+ \alpha^{D}_{f} \lambda^{P} \sinh{\left (L \alpha^{D}_{f} - L_{p} \alpha^{D}_{f} + L_{p} \alpha^{P}_{f} \right )}  \\
& \qquad+ \alpha^{P}_{f} \gamma^{P}_{f} \lambda^{D} \cosh{\left (- L \alpha^{D}_{f} + L_{p} \alpha^{D}_{f} + L_{p} \alpha^{P}_{f} \right )}  \\
& \qquad+ \alpha^{P}_{f} \gamma^{P}_{f} \lambda^{D} \cosh{\left (L \alpha^{D}_{f} - L_{p} \alpha^{D}_{f} + L_{p} \alpha^{P}_{f} \right )}  \\
& \qquad+ \alpha^{P}_{f} \lambda^{D} \sinh{\left (- L \alpha^{D}_{f} + L_{p} \alpha^{D}_{f} + L_{p} \alpha^{P}_{f} \right )}  \\
& \qquad+ \alpha^{P}_{f} \lambda^{D} \sinh{\left (L \alpha^{D}_{f} - L_{p} \alpha^{D}_{f} + L_{p} \alpha^{P}_{f} \right )}
\end{split}
\end{equation}


From this calculus, we can write the PSP at the soma on the form:

\begin{equation}
\label{eq:delta-v-final}
\hat{\delta v}(X=0, X_{src}, f) = K_f(X_{src}) \, 
\big(\mu_v(X_{src})-E_{rev}\big) \,  \hat{g(f)} 
\end{equation}

where \( K_f(X_{src}) \) given by:

\begin{equation}
K_f(X_{src}) = 
\left\{
\begin{split}
& A_f(X_{src}) \quad \forall X_{src} \in [0,L_p] \\
& E_f(X_{src}) \quad \forall X_{src} \in [L_p, L]
\end{split}
\right.
\end{equation}
\end{enumerate}

This is obtained by taking a unitary current \(I_f=1\) in the previous calculus.

\section{Deriving the power spectrum of the membrane potential fluctuations}
\label{sec-4}

The calculus rely on the ability to obtain the power spectrum of the
membrane potential fluctuations at the soma $P_V(f)$.

This can be obtained from shotnoise theory \cite{Daley2007} (see also
\citetext{ElBoustani2009a} for an application similar to ours), the
general form of the power spectrum density can be expressed as:

\begin{equation}
P_V(f)  = \sum_{\{syn\}} N_{syn} \,  F_\mathrm{synch} \, \nu_{syn}
\, \| \hat{\mathrm{PSP}_{syn}}(f) \|^2 
\end{split}
\label{eq:vm-pwd}
\end{equation}

where $\{syn\}$ is the set of identical synapses, each having
$N_{syn}$ synapses, a Poisson release probability: $\nu_{syn}$ and
creating a post-synaptic event $\mathrm{PSP}_{syn}(t)$. In addition,
$F_\mathrm{synch}$ is a synchrony factor (depending on the variable
$s$ in the model), it accounts for the effects of the synchronous
arrivals of presynaptic events. Given the synchrony generator
considered in the main text, the synchrony factor takes the form:

\begin{equation}
\label{eq:F-synch}
F_\mathrm{synch} =(1-s) + (s-s^2) 2^2 + (s^2-s^3) 3^2 + s^3 4^2
\end{split}
\end{equation}

because single events arise with a probability $1-s$, double events
with a probability $s-s^2$ (and the PSP are squared in
Eq. \ref{eq:vm-pwd}, hence the $2^2$ factor), etc...


Now obtaining the power spectrum density $P_V(f)$ in our situation
requires to explicit the sum over synapses: $\sum_{\{syn\}}$, In our
cases, we need to sum over 1) their type (excitatory/inhibitory,
$\sum_{s \in \{e,i\}}$ ), 2) their location (we will integrate over
the dendritic length $\int_0^L dx$) 3) branches.

\begin{equation}
\begin{split}
  P_V(f)  = & \sum_{s \in \{e,i\}} \int_0^L \, dx \,  \pi \, \mathcal{D}_s
  \Big( d_t \, 2^{- \frac{2}{3} \, \lfloor \frac{B \, x}{l} \rfloor} \Big)
  \, \,  2^{\lfloor \frac{B \, x}{l} \rfloor} \, \, \,  F_\mathrm{synch} \, \nu_{s}(x)
  \, \| \hat{\delta v}_{s}(0, x, f) \|^2 \\
  & + \pi \, \mathcal{D}_i \, l_{S} \, d_{S} \, \,  F_\mathrm{synch} \, \nu_{i}(0)
  \, \| \hat{\delta v}_{i}(0, 0, f) \|^2 
\end{split}
\label{eq:vm-pwd-final}
\end{equation}

where $ \hat{\delta v}_{s}(0, x, f)$ is given by
Eq. \ref{eq:delta-v-final} (note that the dependency on synaptic type
$s$ comes from the reversal potential term $E_{rev}$ in
Eq. \ref{eq:delta-v-final}). The factor $2^{\lfloor \frac{B \, x}{l}
  \rfloor}$ corresponds to the sum of the synapses over the different
branches at the distance $x$. The term $\big( d_t \, 2^{- \frac{2}{3}
  \, \lfloor \frac{B \, x}{l} \rfloor} \Big)$ is the diameter of the
branches at the distance $x$.

The last term in Eq. \ref{eq:vm-pwd-final} corresponds to the
contribution of somatic inhibitory synapses (number of somatic
inhibitory synapses: $\pi \, \mathcal{D}_i \, l_{S} \, d_{S}$.

\section{Deriving the fluctuations properties ($\mu_V, \sigma_V, \tau_V$)}
\label{sec-5}

The final expressions for the fluctuation propeorties as a function of
($\nu_e^p, \nu_i^p, \nu_e^d, \nu_i^d, s$) are thus given by:

\begin{itemize}

  \item $\mu_V$: we obtain the mean of the fluctuations at the soma by
    taking $\mu_V(0)$ in Equation \ref{eq:muV-final}.

  \item $\sigma_V$: we obtain the standard deviation of the fluctuations from the power spectrum density in Equation \ref{eq:vm-pwd-final} and the expression:

    \begin{equation}
      \label{eq:var-expr}
      \sigma_V^2 = \int_\mathbb{R} P_V(f) \, df
    \end{equation}

    This integral expression was discretized and evaluated numerically

  \item $\tau_V$: we obtain the autocorrelation time of the
    fluctuations from the power spectrum density in Equation
    \ref{eq:vm-pwd-final} and the expression \cite{Zerlaut2016}:

    \begin{equation}
      \label{eq:Tv-def-with-pwd}
      \tau_V = \frac{1}{2} \, \big( \frac{\int_\mathbb{R} P_V(f) \, d f}{ P_V(0) } \big)^{-1}
    \end{equation}
    
    This integral expression was discretized and evaluated numerically
    
\end{itemize}


\section{References}
\label{sec-5}
\bibliography{biblio}

\end{document}
